\subsection{Basic History}
    In 1959, \emph{Edward F. Moore} presented one of the first shortest path through a maze algorithm
    \cite{MooreRef}, after a couple of years in 1961 \emph{Lee, C. Y.} presented the idea of simulating the 
    board wiring on electronics board as a maze \cite{LeeRef}. Afterwards, the idea of \emph{Lee Maze}
    has been revisited many times. In 1983, \emph{Hightower, D.} made more contribution to the idea, such that
    using modern computers and virtual memories, we can mimic the routing problem precisely providing 
    different techniques \cite{HightowerRef}.

\subsection{Router Problem Anatomy}
    In this paper, we are mainly concerned with \emph{detailed routing}. However, routing is divided into many subsections \cite{routingBook}, as shown in fig.\ref{fig:routing_anat}. We only consider maze and line search \emph{(global routing)}.

    \begin{figure}
        \centering
        \includegraphics[width=0.4\textwidth]{figures/routing_anatomy.png}
        \caption{Anatomy of various routing techniques}
        \label{fig:routing_anat}
    \end{figure}

    In the next subsection, we show how various algorithms work, and discuss three 
    algorithms. First, the \nameref{LeeSection} which is a maze-based algorithm. Second, the 
    \nameref{MikamiSection}, which is a line-probe algorithm. Finally, the \nameref{SteinerSection},
    which is a baseline algorithm.

\subsection{Simulation Environment}
    In brief, we describe the simulation properties:
    \begin{itemize}
        \item Detailed routing \emph{(not global)}.
        \item System is presented, as $3D$ Grid,$[D, W, H]$.
        \begin{itemize}
            \item D: number of layers 
            \item W: width of each grid
            \item H: height of each grid
        \end{itemize}
        \item One source exists as a starting point.
        \item Multiple targets exist \emph{(fan-out)}.          
        \item Consistent Cost, which means no \emph{bending} cost and no \emph{extra} cost due to Vias.
        \item Vias cost is $1$.
        \item Within the same $2D$ grid, path can move \emph{horizontally} or/and \emph{vertically}.
    \end{itemize}

\subsection{Explored Techniques}
    \subsubsection{\underline{Lee algorithm}}
    \label{LeeSection}
    This is one of the most common routing algorithms \cite{LeeRef}.
    If there's a path between source $S$ and some target $T$, the algorithm \emph{definitely} finds it,
    and in case of consistent cost (i.e. no variable cost), the algorithm guarantees the shortest path.
    It uses BFS (breadth-first search) to connect targets with the source.
    It works appropriately with multiple layers (i.e. where Vias exist).
    The algorithm has three main stages:
    \begin{itemize}
        \item Expansion.
        \item Backtracking.
        \item Clean up.
    \end{itemize}

    The \emph{expansion} stage creates a \emph{halo-like} shape around the source, which expands and the cost of each cell is incremented, unless there's an obstacle (block cell). 
    This continues until it hits a target and terminates. If the expansion reaches its limit with no target hit, this means that the target is not reachable.
    
    \begin{figure}
    \label{fig:expansionStage}
    \centering
    \begin{subfigure}[b]{0.3\linewidth}
        \includegraphics[width=\linewidth]{figures/Lee Stages/grid.png}
        \caption{itr 1}
    \end{subfigure}
    \begin{subfigure}[b]{0.3\linewidth}
        \includegraphics[width=\linewidth]{figures/Lee Stages/grid 1.png}
        \caption{itr 2}
    \end{subfigure}
    \begin{subfigure}[b]{0.3\linewidth}
        \includegraphics[width=\linewidth]{figures/Lee Stages/grid 2.png}
        \caption{itr 3}
    \end{subfigure}
    \begin{subfigure}[b]{0.3\linewidth}
        \includegraphics[width=\linewidth]{figures/Lee Stages/grid 3.png}
        \caption{itr 4}
    \end{subfigure}
    \begin{subfigure}[b]{0.3\linewidth}
        \includegraphics[width=\linewidth]{figures/Lee Stages/grid 4.png}
        \caption{itr 5}
    \end{subfigure}
    \begin{subfigure}[b]{0.3\linewidth}
        \includegraphics[width=\linewidth]{figures/Lee Stages/grid 5.png}
        \caption{itr 6}
    \end{subfigure}
    \begin{subfigure}[b]{0.3\linewidth}
        \includegraphics[width=\linewidth]{figures/Lee Stages/grid 6.png}
        \caption{itr 7}
    \end{subfigure}
    \begin{subfigure}[b]{0.3\linewidth}
        \includegraphics[width=\linewidth]{figures/Lee Stages/grid 7.png}
        \caption{itr 8}
    \end{subfigure}
    \begin{subfigure}[b]{0.3\linewidth}
        \includegraphics[width=\linewidth]{figures/Lee Stages/grid 8.png}
        \caption{itr 9}
    \end{subfigure}
    \begin{subfigure}[b]{0.3\linewidth}
        \includegraphics[width=\linewidth]{figures/Lee Stages/grid 9.png}
        \caption{itr 10}
    \end{subfigure}
    \caption{Expansion Stage}
  \end{figure}

    The \emph{backtracking} stage, shown in fig.\ref{fig:backtrackingStage}, gets the path from the target $T$ to the source $S$.
    Since consistent cost is considered, then the backtracking stage is not that much of an issue.
    The cost is to be decremented by $1$ from the target, until we reach the source and the cost of the path is its total length.
    \emph{Data structure}, such as \textbf{priority queue}, is used to \emph{pop-up}
    the cell with minimum cost.

    \begin{figure}
        \centering
        \includegraphics[width=0.35\textwidth]{figures/Lee Stages/back-track.png}
        \caption{Backtracking Stage}
        \label{fig:backtrackingStage}
    \end{figure}

    The \emph{clean up} stage, shown in fig.\ref{fig:cleanUpStage}, converts the path from the source to the target into obstacles \emph{(blocks)}, so that no interference between paths exists.

    \begin{figure}
        \centering
        \includegraphics[width=0.35\textwidth]{figures/Lee Stages/clean_up.png}
        \caption{Clean up Stage}
        \label{fig:cleanUpStage}
    \end{figure}

    \subsubsection{\underline{Mikami-Tabuchi algorithm}}
    \label{MikamiSection}
    Mikami-Tabuchi developed the algorithm named after them \cite{mikami1968computer} to address the issues of Lee maze solver, which is the huge requirements of time and space.

Mikami-Tabuchi is simple, fast, low on memory resources but it doesn't guarantee finding the optimal path. It only guarantee finding a path if exists.

Mikami-Tabuchi algorithm is sometimes referred to as \textit{line-probing algorithm} or \textit{line-search algorithm}.

Mikami-Tabuchi basic idea is to extend horizontal and vertical lines/probes from source and target. The algorithm extends those lines until they either hit an edge of the area or an obstacle.

If one of the lines of the source intersected with one of the lines of the target, the algorithm backtracks from the point of intersection to both source and target, and that's the solution.

\begin{figure}
    \centering
    \includegraphics[width=\linewidth]{figures/mikami.png}
    \caption{Mikami Algorithm Illustration \cite{chen2009global}}
    \label{fig:mikamiIllustr}
\end{figure}

Mikami-Tabuchi algorithm wasn't designed with multi-layer in mind. But it could be extended to work with multi-layers by using 3d probes/lines and representing the VIAs as lines that's perpendicular to the layer. If a line in the layer intersected with the VIA line, the algorithm, recursively, extends a probe on the same point but on the other layer.

\begin{algorithm}[h!]
\SetAlgoLined
\KwResult{Some path between source and destination.}
    Let S and T be a pair source and destination respectively\;
    Generate 4 lines (2 horizontal and 2 vertical) passing through S and T\;
    Extend these lines till they hit obstacles or the boundary of the layout\;
    \If{a line generated from S intersects a line generated from T}{
        backtrack to find path\;
        return path\;
    }
    $i \leftarrow 1$\;
    \While{no new lines were created}{
        \ForEach{line L $\in$ level $i-1$}{%
            \ForEach{Point p $\in$ line L}{%
                Generate a perpendicular line L2\;
                Extend line L2 till it hit obstacles or the boundary of the layout or intersects with other line L3, where $L3 \in level j$ and $j > 0$\;
                \If{L2 intersects with L3 and L2 and L3 are from different terminals}{%
                    backtrack to find path\;
                    return path\;
                }
            }
        }
        $i \leftarrow i+1$
    }
    \caption{Mikami-Tabuchi Algorithm for Automatic Routing \cite{mikamiPresent}}
\end{algorithm}

    \subsubsection{\underline{Steiner algorithm}}
    \label{SteinerSection}
    The definition of \emph{Steiner tree} is general, we are mainly concerned with minimum Steiner tree problem, and it's algorithm.
    Minimum Steiner Tree $(MST)$ algorithm provides the minimum spanning tree that may exist in the graph,
    so that it connects multiple points with each others, using intermediate points called \emph{Steiner points}.
    
    It's basic idea is that, after including the first path, shown in fig.\ref{fig:steiner_1}, the shortest/closest target to the source any other target that will be connected to the source may be connected to any point of the 
    extracted path, so that it could be connected to the main source or any Steiner point that's being produced \emph{(Grey cells)},
    and to connect that target, we need to know which of the left targets has the minimum distance to the source and all of the Steiner points, then pick up the one with the minimum cost, as shown in fig.\ref{fig:steiner_2}, and that's what makes time grows exponentially.

    \begin{figure}
        \centering
        \includegraphics[width=0.35\textwidth]{figures/Steiner Stages/steiner_1.png}
        \caption{Steiner tree: step 1}
        \label{fig:steiner_1}
    \end{figure}

    \begin{figure}[H]
        \centering
        \includegraphics[width=0.35\textwidth]{figures/Steiner Stages/steiner_2.png}
        \caption{Steiner tree: step 2}
        \label{fig:steiner_2}
    \end{figure}

    \begin{figure}[H]
        \centering
        \includegraphics[width=0.35\textwidth]{figures/Steiner Stages/steiner_3.png}
        \caption{Steiner tree: step 3}
        \label{fig:steiner_3}
    \end{figure}

    The algorithm works as follows \cite{SteinerRef}:
    
    \begin{algorithm}
        \SetAlgoLined
        \KwResult{Optimal/Minimum path between source and all target cells.}
        Find the closest target to the source and Steiner points\;
        \While{$T$ Doesn't span all terminals}{
        Select terminal $x$ not in $T$ that is closest to a vertex in $T$\;
        Add to $T$ the shortest path that connects $x$ with $T$\;
        }
         \caption{Steiner Tree algorithm For Maze Routing}
    \end{algorithm}
        
    As mentioned before, Steiner tree always provide the optimal path, when it exists,
    but the Steiner tree problem is $NP$ Hard problem, both exact and approximate solutions exist,
    and as for our condition we are interested in the exact solution.
    The solution provided to the problem is very simple, the minimum path to be found 
    every time is provided using \emph{Dijkstra} algorithm, due to all this computations, as the dimensions of the graph get larger and the number
    of targets increase the time complexity increases exponentially.
    The intermediate points that are now sources and targets can be connected to are called
    Steiner points.