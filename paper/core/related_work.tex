\subsection{Basic History}
    In 1959, Moore, Edward F. presented one of the first shortest path through a maze algorithm
    \cite{MooreRef}, after a couple of years in 1961 Lee, C. Y. presented the idea of simulating the 
    board wiring on electronics board as a Maze \cite{LeeRef}. Starting from there the idea of Lee Maze
    has been revisited many times, in 1983 Hightower, D. made more contribution to the idea such that
    using modern computers and virtual memories we can memic the routing problem precisely providing 
    different techniques \cite{HightowerRef}.
\subsection{Router Anatomy}
    The Routing problem has many sections and subsections, in this paper we are mainly concerned with
    Detailed routing. Detailed routing is divided into many subsections, as you can see in fig.\ref{fig:routing_anat} 
    and we are exploring Maze and Line Search subsections.

    \begin{figure}[H]
        \centering
        \includegraphics[width=0.4\textwidth]{figures/routing_anatomy.png}
        \caption{Anatomy of various routing techniques.}
        \label{fig:routing_anat}
    \end{figure}

    In the next section we are showing how various algorithms work, and discussing three 
    algorithms, first the \nameref{LeeSection} it's a Maze based algorithm, second the 
    \nameref{MikamiSection} it's a Line-Probe algorithm, and finally the \nameref{SteinerSection} 
    it's a baseline algorithm.

\subsection{Explored Techniques}
    Before diving into these algorithms make sure to read the \nameref{terminologySection} section first.
    \newline

    \subsubsection{Lee algorithm}
    \label{LeeSection}
    This is one of the most common and origin routing algorithms \cite{LeeRef}.

    If there's a path between source $S$ and some target $T$, the algorithm will definitely find it,
    and in case of consistent cost (ie. no variable cost) the algorithm will not only find the 
    a path but also the shortest one.

    It uses BFS (breadth-first search) to connect targets with the source.

    It works appropriately with multiple layers (ie, where vias exist).

    The algorithm has main three stages:
    \begin{itemize}
        \item Expansion
        \item Back-tracking
        \item Clean up
    \end{itemize}

    The \textit{Expansion} stage fig.\ref{expansionStage} creates like a \textit{halo} shape around the source,
    and it gets larger and the cost of each cell is incremented,
    unless there's an obstacle (block cell),
    until it hits a target and terminates,
    if the expansion reached it's limit with no target hit, this means that the target/s is/are not
    reachable.

    \begin{figure}[htb]
        \label{fig:expansionStage}
        \centering

        \begin{subfigure}[b]{0.3\linewidth}
            \includegraphics[width=\linewidth]{figures/Lee Stages/grid.png}
            \caption{itr 1}
        \end{subfigure}
        \begin{subfigure}[b]{0.3\linewidth}
            \includegraphics[width=\linewidth]{figures/Lee Stages/grid 1.png}
            \caption{itr 2}
        \end{subfigure}
        \begin{subfigure}[b]{0.3\linewidth}
            \includegraphics[width=\linewidth]{figures/Lee Stages/grid 2.png}
            \caption{itr 3}
        \end{subfigure}
        \begin{subfigure}[b]{0.3\linewidth}
            \includegraphics[width=\linewidth]{figures/Lee Stages/grid 3.png}
            \caption{itr 4}
        \end{subfigure}
        \begin{subfigure}[b]{0.3\linewidth}
            \includegraphics[width=\linewidth]{figures/Lee Stages/grid 4.png}
            \caption{itr 5}
        \end{subfigure}
        \begin{subfigure}[b]{0.3\linewidth}
            \includegraphics[width=\linewidth]{figures/Lee Stages/grid 5.png}
            \caption{itr 6}
        \end{subfigure}
        \begin{subfigure}[b]{0.3\linewidth}
            \includegraphics[width=\linewidth]{figures/Lee Stages/grid 6.png}
            \caption{itr 7}
        \end{subfigure}
        \begin{subfigure}[b]{0.3\linewidth}
            \includegraphics[width=\linewidth]{figures/Lee Stages/grid 7.png}
            \caption{itr 8}
        \end{subfigure}
        \begin{subfigure}[b]{0.3\linewidth}
            \includegraphics[width=\linewidth]{figures/Lee Stages/grid 8.png}
            \caption{itr 9}
        \end{subfigure}
        \begin{subfigure}[b]{0.3\linewidth}
            \includegraphics[width=\linewidth]{figures/Lee Stages/grid 9.png}
            \caption{itr 10}
        \end{subfigure}
        
        \caption{Expansion Stage}
      \end{figure}
      

    The \textit{Back-tracking} stage fig.\ref{fig:backtrackingStage} gets the path from the target $T$
    to the source $S$.
    Since we are dealing with consistent cost, then the backtracking stage is not that much of an issue
    all we have to do is to decrement the cost by $1$ from the target, until we reach the source.
    In other versions where inconsistent cost exist, and the cost of the path is the total length of it.
    \textbf{Data structure} such as \textbf{priority queue} is used to \textit{pop-up}
    the cell with minimum cost.

    \begin{figure}[H]
        \centering
        \includegraphics[width=0.35\textwidth]{figures/Lee Stages/back-track.png}
        \caption{Back-tracking Stage}
        \label{fig:backtrackingStage}
    \end{figure}

    The \textit{Clean up} stage fig.\ref{fig:cleanUpStage} converts the path from the source
    to the target into obstacles (blocks) so that no interference between pathes may exist.
    and then starts to connect another target to the source with that path of obstacles added.

    \begin{figure}[H]
        \centering
        \includegraphics[width=0.35\textwidth]{figures/Lee Stages/clean_up.png}
        \caption{Clean up Stage}
        \label{fig:cleanUpStage}
    \end{figure}

    \subsubsection{Mikami-Tabuchi algorithm}
    \label{MikamiSection}
    \subsubsection{Steiner algorithm}
    \label{SteinerSection}
    
