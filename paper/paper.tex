\documentclass[conference,11pt]{IEEEtran}
\IEEEoverridecommandlockouts
\usepackage[verbose,expansion=alltext,stretch=50]{microtype}
\usepackage{graphicx}
\usepackage{booktabs}
\usepackage[hidelinks]{hyperref}
\usepackage[ruled,vlined]{algorithm2e}
\usepackage{xcolor}
\usepackage{float}
\usepackage{subcaption}
\usepackage[labelformat=parens,labelsep=quad,skip=3pt]{caption}

\newcommand{\link}[1]{{\color{blue}\href{#1}{#1}}}
\title{Exploring A* Search For Single and Multi Layer Routing}

\author{
    Team \#1\\
    \IEEEauthorblockN{
        Mohamed Shawky \small\texttt{SEC:2, BN:16}\IEEEauthorrefmark{1},
        Remonda Talaat \small\texttt{SEC:1, BN:20}\IEEEauthorrefmark{2},\\ 
        Mahmoud Othman Adas \small\texttt{SEC:2, BN:21}\IEEEauthorrefmark{3} and
        Evram Youssef \small\texttt{SEC:1, BN:9}\IEEEauthorrefmark{4}}
    \IEEEauthorblockA{
        \\Department of Computer Engineering,
        Cairo University\\
        Email: 
        \IEEEauthorrefmark{1}mohamed.sabae99@eng-st.cu.edu.eg,
        \IEEEauthorrefmark{2}Remonda.Bastawres99@eng-st.cu.edu.eg,\\
        \IEEEauthorrefmark{3}mahmoud.ibrahim97@eng-st.cu.edu.eg,
        \IEEEauthorrefmark{4}evram.narouz00@eng-st.cu.edu.eg
    }
}

\begin{document}
\maketitle

\begin{abstract}
The recent advances in supercomputers and massive data centres increase the demand for fast processing on various types of data. Consequently, high performance chips become one of the critical factors in any computational system. Modern chips can contain million or even billions of transistors to achieve high performance and demanded functionalities. This makes the process of their design, implementation and integration very tedious. One of the key challenges in modern chip manufacture is routing. Routing is the process of wiring different source transistors to their fan-outs. This process can be very complicated in huge chips and requires the usage of specialized software, which is known as automatic routing. Several research projects have worked on this problem and various techniques are proposed. These techniques are mainly a compromise between the execution time and the solution optimality. In this work, we propose the usage of A* search algorithm, with a simple modified cost function, to solve the routing problem in a grid search formulation. The proposed solution can achieve the same or even better results than the considered baselines, while maintaining a reasonable execution time.
\end{abstract}

\begin{IEEEkeywords}
vlsi, routing, grid search, a* search, maze, steiner tree, mikami-tabuchi
\end{IEEEkeywords}

\section{Introduction}
Fast running systems requires many transistors, which makes the manufacturing process very hard.
Many Chips have millions or even billions of transistors which affects the circuit timing, power consumption, chip reliability and manufacturability that complicate all the design rules.
One of the most important challenge is routing to connect the transistors, without causing any problem on the chip.
Routing problem, in \emph{VLSI}, is considered an \emph{NP-hard} problem, so it is divided into two design phases, \emph{global routing} where the grid is constructed with its nodes and edges,
and \emph{detailed phase} (our target) to find shortest paths to connect the required pins in the grid together.
Many algorithms can be used to solve routing problem, as it can be formulated as grid search problem, where speedup and optimality are a trade-off.
We try \emph{Lee's}, \emph{Mikami-Tabuchi}, \emph{Steiner tree} and \emph{A* search} algorithms, where our main approach was A* search.
We compare the different algorithms based on their execution time and length of metal.



We discussed the Detailed Routing problem, and shows that it can be formulated as gird search problem.
A* algorithm with euclidean distance as cost function was our main approach in the problem solution.
We tried different gird search techniques and compared their results with our main approach with respect to performance and time, and found that
A* achieves the better performance in some cases and same performance in the others in acceptable time compared to other algorithms.

\section{Terminology}
\subsection{Basics} %There's a space here, don't erase it.
\begin{itemize}
	\item Maze: it's a $[D,W,H]$ matrix that contains cells, used to simulate the grid.
    \item D: number of layers, W: width of the layer, H: height of the layer.
    \item Cells: each cell is a pin, a cell could represent a pin or an obstalce.
	\item Pin: the part where transistors gets connected to.
	\item Source: the starting point (pin) that needs to be connected to some targets.
	\item Tatgets: one or many point/s (pin/s) that need to be connected to the source in minimum cost.
	\item Obstacle: a block that wires can't go through.
	\item Vias: like a ladder to the upper or lower layer.
	\item Wire: what connects the pins with each others.
	\item Path: the route which the wire will take in order to connect the source with all of the targets.
	\item Cost: the length of the path, the longer the wire the larger the delay.
	\item Multi-layers: instead of having only one $2D$ grid, we have multiple grids, stacked vertically.
	\item Steiner Point: intermediate points that targets can be connected to.
\end{itemize}

\subsection{Assumptions} 
\begin{itemize}
    \item All wires are the same size.
    \item No geometric rules violations(ie. spacing).
    \item All pins are placed at the center of cells.
\end{itemize}

\label{terminologySection}

\section{Related Work}
\subsection{Basic History}
    In 1959, Moore, Edward F. presented one of the first shortest path through a maze algorithm
    \cite{MooreRef}, after a couple of years in 1961 Lee, C. Y. presented the idea of simulating the 
    board wiring on electronics board as a Maze \cite{LeeRef}. Starting from there the idea of Lee Maze
    has been revisited many times, in 1983 Hightower, D. made more contribution to the idea such that
    using modern computers and virtual memories we can memic the routing problem precisely providing 
    different techniques \cite{HightowerRef}.
\subsection{Router Anatomy}
    The Routing problem has many sections and subsections, in this paper we are mainly concerned with
    Detailed routing. Detailed routing is divided into many subsections, as you can see in fig 
    \ref{fig:routing_anat} and we are exploring Maze and Line Search subsections.

    \begin{figure}[H]
        \centering
        \includegraphics[width=0.4\textwidth]{figures/routing_anatomy.png}
        \caption{Anatomy of various routing techniques.}
        \label{fig:routing_anat}
    \end{figure}

    In the next section we are showing how various algorithms work, and discussing three 
    algorithms, first the \nameref{LeeSection} it's a Maze based algorithm, second the 
    \nameref{MikamiSection} it's a Line-Probe algorithm, and finally the \nameref{SteinerSection} 
    it's a baseline algorithm.

\subsection{Explored Techniques}
    \subsubsection{Lee algorithm}
    \label{LeeSection}
    \subsubsection{Mikami algorithm}
    \label{MikamiSection}
    \subsubsection{Steiner algorithm}
    \label{SteinerSection}
    


\section{Methodology}
\subsection{Motivation}
Metal routing is a critical step in systems integration process, where each source is connected to its fan-outs using non-crossing metal. Routing in a modern chip, with millions or even billions of transistors, can be very complicated and cumbersome, so the manufacture process has moved to automatic routing. 

Automatic routing is a very vast field, with several techniques being developed through time. Automatic routing involves lots of problems like finding shortest non-blocked path and VIAs in multi-layer routing. 

Automatic routing techniques are always a trade-off between optimality and speed. Some techniques target execution time by reaching a sub-optimal solution. Others target optimal solution with very high execution time.

In this work, we seek a good balance between performance and optimality of solution by introducing the usage of A* search for automatic routing with a modified cost function.

\subsection{Formulation}
Routing is formulated as a grid search problem, where an occupancy grid map \emph{(OGM)} is provided with some source and destination cells to be joined. The \emph{OGM} cells can have the value of $1$ for occupied cell or $0$ for empty cell. The \emph{OGM} can be of multiple levels, in case of multi-layer routing, where the cell can have a third value to indicate that the cell can be used as a VIA. 

Given that problem formulation, several grid search and shortest path algorithms can be used to find the optimal path between each source and its given destinations. These algorithms can vary based on optimality and performance.

The two main metrics considered in this work are path length and execution time. Path length between each source and destination cell is used as a measure of solution optimality. It's defined as the number of grid cells that lies on the deduced path.

Execution time is used as a measure of algorithm performance. It's simply the total elapsed time by the algorithm to find all required paths.

\subsection{Baseline}
To measure the validity and performance of our proposed method, three main baselines are used. These baselines are well-established algorithms that are currently used in industry.

\subsubsection{Maze Routing (Lee's Algorithm)}

\subsubsection{Mikami-Tabuchi’s Algorithm}

\subsubsection{TBD}

\subsection{Modified A* Search}

\section{Experimental Results}
\subsection{Goals}
We want to answer those questions about our proposed algorithm:
\begin{enumerate}
    \item How fast is it?
    \item How short are the found paths?
\end{enumerate}
Because \textit{fast} and \textit{short} are relative terms. We need to calculate them in a comparison with other approaches. 
We chose to compare our results aginst:
\begin{itemize}
    \item Mikami: known by its speed, but very far from the optimal answer. Would give us insights on how much speed we have achieved.
    \item Steiner Tree: Finds the optimal answer. We will use it to know whether our results are optimal.
    \item Lee Maze: First algorithm used in the industry. Helps us know how far we have progressed relative to the industry practical techniques.
\end{itemize}

\subsection{Code}
We implmented our proposd algorithm in \texttt{mod\_a\_star.py}, 
lee maze algorithm in \texttt{maze\_lee.py},
mikami-tabuchi algorithm in \texttt{mikami\_tabuchi.py}
and steiner tree in \texttt{steiner\_tree.py}.

All scripts read json input from the \texttt{stdin} and writes json output to \texttt{stdout}, so we can chain them with the other scripts. And they follow the \texttt{io\_schema.md} specs about io format.

\subsection{IO Specs}
Input contains a $d\times h \times w$ grid matrix, where:
\begin{itemize}
    \item $d \rightarrow$ Number of layers, either 1 or 2.
    \item $h \rightarrow$ Hieght.
    \item $w \rightarrow$ Width.
\end{itemize}
Each cell is either:
\begin{itemize}
    \item 0 $\rightarrow$ Empty.
    \item 1 $\rightarrow$ Obstacle.
    \item 2 $\rightarrow$ VIA, only when $d=2$ and must exist on the other layer too.
\end{itemize}
Each input contains the source coordinates and a list of targets coordinates.

Output should contain the found paths and their corresponding lengthes.

Both input and output should be in json format. See \texttt{io\_schema.md} for more details.

\subsection{Helper Scripts}
We wrote a couple of scripts to assist with the comparison and testing:
\begin{itemize}
    \item \texttt{gen-input.py}: Generates random input that follows \texttt{io\_schema.md}. It doesn't gurantee that all targets are reachable. For more info \texttt{\$ python3 gen-input.py --help}
    \item \texttt{verify.py}: Takes the input to the algorithm and its output and verifies the correctness of the result. See \texttt{\$ python3 verify.py --help}
    \item \texttt{random\_test}: Generates infinite random inputs, run given algorithm on each test, and verify the results.
    \item \texttt{random\_comp}: Generates $N$ random input, run each algorithm on each input, calls \texttt{calc\_total.py} to calculate total cost and verify the results.
    \item \texttt{nConst}: Calls \texttt{random\_comp} $M$ times, each time with same number of targets, varying the grid area (w,h).
    \item \texttt{areaConst}: Calls \texttt{random\_comp} $M$ times, each time with same width and height of the input grid, varying the number of targets.
    \item \texttt{merge\_comp}: Merges the outputs of \texttt{random\_comp} into one \texttt{tmp/summary.json} with the summary of the experiment.
    \item \texttt{plot.py}: Plots given \texttt{summary.json} through the stdin to \texttt{tmp/plot}.
\end{itemize}

\subsection{The Experiment}
For simplicity, we assumed all grids are squares. So $w=h$ in all tests.
We also assumed number of layers $d=2$ in all tests.

We need to very the area while the number of targets is constant. And in the other case, vary the number of targets while the area is constant.
And in both, we will record and plot the running times and costs.

We expect some algorithm to take a very big time to calculate the output. Unfortunately we can't just let it run forever. So we just sat the timeout as 5 minutes. This is why we collected the number of found targets, so we compare how many times an implmentation has timeouted and resulted in 0 final targets found.

\textbf{Note:} Not all targets in some input have to be reachable. Bigger grids have bigger probability of having non-reachable targets.

We started the experiments by running:
\begin{enumerate}
    \item \texttt{nConst}, which for each area of areas of grid in [10, 15, 20, 50, 100], conducts 5 random experiments on each algo given the same random input (for each experiment) in which the number of targets ($n$) is const and $n = 5$.
    \item \texttt{areaConst}, which for each number of targets in [6, 10, 15, 20, 50], conducts 20 random experiments on each algo given the same random input (for each experiment) in which the area of the grid ($w,h$) is constant and $w = h = 45$.
\end{enumerate}

After running the 2 scripts multiple times and then merging their outputs using \texttt{merge\_comp}, we had so far 396 unique experiment, each experiment is a unique input given to the 4 algorithms and all the 1584 results are in \texttt{summary.json}.

\subsection{Comparisons}
We made the plots using \texttt{plot.py}. The following is the line of thoughts we had through the comparison.

\subsubsection{Running Time Scatter}
We started by scaterring all the running time per \#Targets and per Grid Width, see Fig. (\ref{fig:runningTime}). 

The results are not very clear as the points of mikami and lee are squished down becaus of stein and a*. 

We can also notice a* has some outliers in time, specially in bigger grid widths (50, 100). This is why we may use the median function multiple times instead of the average.

\begin{figure}
\centering

\begin{subfigure}[b]{\linewidth}
    \includegraphics[width=\linewidth]{figures/plots/areaConst.png}
    \caption{Time / \#Targets}
\end{subfigure}
\begin{subfigure}[b]{\linewidth}
    \includegraphics[width=\linewidth]{figures/plots/nConst.png}
    \caption{Time / Grid Width}
\end{subfigure}

\caption{Running Time}
\label{fig:runningTime}
\end{figure}

\section{Conclusion}
In this work, we discuss the \emph{ Detailed Routing } problem, and how it's important challenge in manufacturing process, showing that it can be formulated as gird search problem.
\emph{A* Search} algorithm, with euclidean distance as cost function, is our main approach in the problem solution.
We have tried different gird search techniques and compared their results with our main approach with respect to performance and time, and found that
\emph{A* Search} achieves better performance in some cases and same performance in others, in acceptable time, compared to other algorithms.

\medskip

\bibliographystyle{unsrt}
\bibliography{paper}
    
\end{document}
